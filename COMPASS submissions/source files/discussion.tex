\section{Discussion}

The use of WhatsApp in India has been transcending class boundaries~\cite{mint_2018_how}, motivating us to explore whether its API can be used for distributing and crowdsourcing stories from communities that may not even be able to read or write. Our demo enabled users to contribute content in video, voice and vernacular- 3Vs that have been central to the growth of Internet users in India~\cite{google_2020_3V}. In particular, augmenting the phone numbers of existing IVR platforms with the WhatsApp API to allow users to submit videos (as opposed to only audio) has immense potential in scaling up the voices of marginalized communities and promoting linguistic diversity to create a more inclusive Internet. However, this model relies upon an ecosystem of cheap  mobile data, since users use their own bandwidth for submissions via WhatsApp unlike an IVR platform that relies upon a simple missed call.

Another community media platform making extensive use of WhatsApp is Khabar Lahariya~\cite{sinha_2018_reimagining}, whose reporters use WhatsApp to send videos to a central editorial team for processing and dissemination across multiple platforms and news organizations. We would argue that use of the WhatsApp API, which allows for providing acknowledgements, explanations, instructions and updates through automatic replies, is more accessible to people at the grassroots who want to be journalists and report their own stories. Compared to using a regular WhatsApp account for receiving stories, the API allows for a more robust two-way communication with grassroots communities that can be used to distribute stories back to them and conduct short polls, surveys and interviews. At the same time, expanding the base of reporters requires clear disclosure policies, as many may not be aware that stories they report are uploaded with their name against it and will remain on the internet for posterity.

Finally, an emerging body of research speaks to the benefit of `meeting people where they are' or integrating with existing platforms already in use, rather than creating unfamiliar, custom-built platform \cite{lambton2020unplatformed, saldivar2019online}. CGNet's earlier attempts to crowdsource and distribute stories through dedicated smartphone applications floundered due to the training and installation overheads for onboarding new users \cite{d2014mobile, mehta2020facilitating}. By contrast, the CGNet field team now simply instructs new users to call the IVR number and report their stories if they do not have internet, or to WhatsApp them to the same number if they have a smartphone. If there is one takeaway we hope researchers and practitioners will absorb from this short paper, it is this: why build an app, when you can use WhatsApp?

% \section{Acknowledgements}

% We would like to thank Aditya Vashistha and Amit Sharma for their comments on an early outline of this paper and Shubhranshu Choudhary at CGNet Swara for his participation in the design process. We are also grateful to the Center for Societal Impact through Cloud and AI (SCAI) at Microsoft Research India for their steadfast support over the years.

%Through this demo, we demonstrate how IVR based platforms can evolve as more and more people get access to the internet and smartphones. . This solution is one step above earlier efforts as WhatsApp offers richer media over the traditional IVR system, which is also challenging to scale due to cost; no training/installation is required, unlike in other smartphone apps based interventions. WhatsApp also caters to users with intermittent internet access, as users can report stories without the Internet which then get submitted whenever they get to access the internet. 
% Both of these extensions are truly empowering for emerging users, who have limited familiarity with interfaces such as IVR and special-purpose apps, but are increasingly familiar and comfortable with WhatsApp.


 %In contrast the organisation X trains and enables people to do their own reporting. While the organisation X has to pay the fixed cost of $\$500$ per month for hosting the WhatsApp Business API, it has its advantages of offering a friendly chat-like feeling to users through . Apart from its use for citizen journalism, the WhatsApp channel can be further used to conduct short polls, surveys, interviews for collecting grass-root level data. It would be of interest to replicate studies like Learn2Earn~\cite{swaminathan_2019_learn2earn} by leveraging the richer media content.


%In future, the dissemination of stories can be improved by developing personalised recommendation system. The organisation also plans to customise the user experience by providing the functionality to enable users to request locally relevant stories by asking them to share their location via the `Share current location' feature of WhatsApp. Another design feature in development is- `Moderator Mode', for provision of differentiating the stories submitted by users from stories submitted by reporter for a rural user without a smartphone. This mode would be beneficial for the organisation to keep track of and design incentive schemes based on progress of reporters, e.g., stories submitted by a moderator per day, and for the on-field moderators/reporters, e.g., they can type in metadata, such as its title and description to reduce the work of the backend moderation team.

% While the main purpose of the 'Moderator Mode' has been to differentiate the stories submitted by users from stories submitted by moderators for someone else, in future, the backend program for this mode can be programmed to automatically upload the metadata received by moderators to the loudblog website. This mode would also be helpful for the organisation in future to keep track of progress of moderators, e.g., stories submitted by a moderator per day and design incentive schemes for them enable maximum stories submission. 
% Users not owning a smartphone, can have their story reported by reporters on field visits, who can access a special 'Moderator' mode. This feature specifically helps to distinguish between the reporter and owner of the story, i.e, through this feature the owner of story can be identified even if his/her story has been submitted by a reported. Additionally, reporters can type in metadata, such as its title and description to reduce the work of the backend moderation team. This metadata could then be manually typed in the loudblog website. 

%Users not owning a smartphone, can have their story reported by reporters on field visits, who can access a special 'Moderator' mode. This feature specifically helps to distinguish between the reporter and owner of the story, i.e, through this feature the owner of story can be identified even if his/her story has been submitted by a reported. Additionally, reporters can type in metadata, such as its title and description to reduce the work of the backend moderation team. This metadata could then be manually typed in the loudblog website. 


%growth of internet, initial software companies that built operating systems (microsoft, apple), then in the social web came companies that built on top of operating systems (amazon, google, facebook) and we now believe we are heading into the networked web where next big internet companies will build off the companies that became big in the social web, similar to how those companies built off the ones that created operating systems. can see advantages of unplatforming very clearly, no app or installation required. some countries like china are already ahead of the curve, where new businesses only opt for wechat page and dont even have their own platform or website.

%importance of having a hybrid model, if only build a platform for low network users, only those will end up using it as people with higher connectivity will migrate to those other platforms. important to evolve with a community while still ensuring no one is left behind, ahve tried to do that with hybrid where people can either call if no internet or whatsapp if they have a smartphone and connectivity. focus on just bottom of the pyramid without the layers above will pose sustainability problems among other issues.

%sustainability for the organization, ivr unable to scale while this one can. it is also limited in its functionality, unlike whatsapp.
