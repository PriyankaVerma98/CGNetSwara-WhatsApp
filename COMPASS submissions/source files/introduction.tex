\section{Introduction}

Online communication platforms and the mushrooming of citizen journalists have fundamentally changed our world, from Facebook groups that helped organize mass protests in Tunisia and Egypt to YouTube videos of citizens under fire from government forces in Syria. These changes have rendered an operational shift in the journalist's duty of keeping the public informed, from being a gatekeeper of information to a custodian of information who must curate, verify and lend credibility and context to content already in the public domain. Bruns~\cite{bruns_2008_active} highlights the role of journalists as shifting from “gatekeeping” to “gatewatching”, while Dailey and Starbird express the changing role of journalists by describing them as crowdsourcerers~\cite{starbird_2014_journalists} who ‘incorporate the crowd as co-collaborators’ to ‘collect, curate, synthesise, and re-broadcast information across technological divides.’

At the same time, this democratization of the information space is occuring unevenly across the world, due to varying levels of technical proficiency, infrastructural barriers and high literacy requirements for producing content on most social media platforms. Nonetheless, the smartphone revolution and its increasing penetration among the next billion users of the internet are increasing the reach of citizen journalism platforms. The intuitive interfaces on apps like Tik Tok, YouTube and WhatsApp have allowed even semi-literate users to make extensive use of these platforms, thus opening up a new design space for letting these communities become citizen journalists who can tell their own stories.

In this paper, we showcase a demo built for CGNet Swara, a citizen journalism platform based out of Central India. For over a decade, CGNet has operated an interactive voice response (IVR) platform that allows users to report or listen to stories by simply calling a toll-free number. Upon seeing that many users now have WhatsApp, we integrated their IVR number with the WhatsApp Business API, thus allowing user generated content submitted via WhatsApp to be posted on YouTube, Facebook, Twitter and CGNet's website, similar to how it is done with their IVR channel. The WhatsApp channel has been designed to supplement and not substitute for IVR, which remains an important medium for engaging communities too poor to own a smartphone, too low literate to navigate a visual interface or too remote to access the internet \cite{vashistha2019social}. We hope to contribute to the research community by both sharing our demo and the challenges we faced in designing with the WhatsApp API, and stimulating conversations around how low-resource communities can join the citizen journalism movement and share their stories with the world.


%In recent times, there has been a disruption in traditional models of journalism with the rise of information and communication technology, mainly social media platforms. The emerging paradigm of journalism- citizen journalism~\cite{gillmor_2006_we}, includes citizens as producers as well as consumers of information. Starbird expressed the changing role of journalists as crowdsourcerers~\cite{starbird_2014_journalists}, to ‘incorporate the crowd as co-collaborators’ to ‘collect, curate, synthesise, and re-broadcast information across technological divides’. Similarly, Bruns~\cite{bruns_2008_active} highlights the role of journalists to be shifting from “gatekeeping” to “gatewatching”, which requires harnessing the collective intelligence and knowledge of dedicated communities and not in ownership or control of information.

%However, not all individuals in the social pyramid have access to mainstream media outlets or social media due to technical proficiency, infrastructural and literacy barriers. However, there is now a growing number of internet users from rural India (18\% in 2019~\cite{iamai_2019_digital}) because of affordability of mobile data (Jio Effect) and rising smartphone penetration. Almost all (~99\%) rural users in India access the internet from their mobile devices and activities like social networking/chat apps and entertainment contribute to most (~80\%) of their usage of the internet~\cite{iamai_2019_digital}. Smartphones and mobile internet offer a futuristic scope for disseminating information, unlike telecenters, information kiosks based interventions, which have not proven very successful in the past~\cite{toyama_2018_from}. Hence, mobile apps like WhatsApp, YouTube open up an opportunity to act as bridge technologies to enable the rural population to contribute to citizen journalism. WhatsApp alone has 400 million users in India~\cite{iamai_2019_digital}.

%Hence, we tap the opportunity to democratise media at the grass-roots level using a chat like friendly interface. We deployed a first-of-its-kind citizen journalism platform using the WhatsApp Business API, which helps take local voices to the global level  by posting user generated content submitted via WhatsApp on YouTube, Facebook, Twitter and the organization's website. Through the technical intervention, we enable smartphone owners to access, report, and share citizen journalism reports with others. It also aids in the financial sustainability (i.e., by avoiding call costs) of the organisation. We contribute to the research community by sharing our design ideas, deployment challenges, and enabling low-resource communities to share their stories easily with global outreach for their stories.
